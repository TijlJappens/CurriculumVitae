%%%%%%%%%%%%%%%%%%%%%%%%%%%%%%%%%%%%%%%%%%%%%%%%%%%%%%%%%%%%%%%%%%%%%%
% LaTeX Template: Curriculum Vitae
%
% Source: http://www.howtotex.com/
% Feel free to distribute this template, but please keep the
% referal to HowToTeX.com.
% Date: July 2011
% 
%%%%%%%%%%%%%%%%%%%%%%%%%%%%%%%%%%%%%%%%%%%%%%%%%%%%%%%%%%%%%%%%%%%%%%
% How to use writeLaTeX: 
%
% You edit the source code here on the left, and the preview on the
% right shows you the result within a few seconds.
%
% Bookmark this page and share the URL with your co-authors. They can
% edit at the same time!
%
% You can upload figures, bibliographies, custom classes and
% styles using the files menu.
%
% If you're new to LaTeX, the wikibook is a great place to start:
% http://en.wikibooks.org/wiki/LaTeX
%
%%%%%%%%%%%%%%%%%%%%%%%%%%%%%%%%%%%%%%%%%%%%%%%%%%%%%%%%%%%%%%%%%%%%%%
\documentclass[paper=a4,fontsize=11pt]{scrartcl} % KOMA-article class
							
\usepackage[english]{babel}
\usepackage{verbatim}
\usepackage[utf8x]{inputenc}
\usepackage[protrusion=true,expansion=true]{microtype}
\usepackage{amsmath,amsfonts,amsthm}     % Math packages
\usepackage{graphicx}                    % Enable pdflatex
\usepackage[svgnames]{xcolor}            % Colors by their 'svgnames'
\usepackage{geometry}
	\textheight=700px                    % Saving trees ;-)
\usepackage{url}
\usepackage{hyperref}

\frenchspacing              % Better looking spacings after periods
\pagestyle{empty}           % No pagenumbers/headers/footers

%%% Custom sectioning (sectsty package)
%%% ------------------------------------------------------------
\usepackage{sectsty}

\sectionfont{%			            % Change font of \section command
	\usefont{OT1}{phv}{b}{n}%		% bch-b-n: CharterBT-Bold font
	\sectionrule{0pt}{0pt}{-5pt}{3pt}}

%%% Macros
%%% ------------------------------------------------------------
\newlength{\spacebox}
\settowidth{\spacebox}{8888888888}			% Box to align text
\newcommand{\sepspace}{\vspace*{1em}}		% Vertical space macro

\newcommand{\MyName}[1]{ % Name
		\Huge \usefont{OT1}{phv}{b}{n} \hfill #1
		\par \normalsize \normalfont}
		
\newcommand{\MySlogan}[1]{ % Slogan (optional)
		\large \usefont{OT1}{phv}{m}{n}\hfill \textit{#1}
		\par \normalsize \normalfont}

\newcommand{\NewPart}[1]{\section*{\uppercase{#1}}}

\newcommand{\PersonalEntry}[2]{
		\noindent\hangindent=2em\hangafter=0 % Indentation
		\parbox{\spacebox}{        % Box to align text
		\textit{#1}}		       % Entry name (birth, address, etc.)
		\hspace{1.5em} #2 \par}    % Entry value

\newcommand{\SkillsEntry}[2]{      % Same as \PersonalEntry
		\noindent\hangindent=2em\hangafter=0 % Indentation
		\parbox[t]{\spacebox}{        % Box to align text
		\textit{#1}}			   % Entry name (birth, address, etc.)
		\hspace{1.5em} \parbox[t]{0.8\textwidth}{#2} \par}    % Entry value	
		
\newcommand{\EducationEntry}[4]{
		\noindent \textbf{#1} \hfill      % Study
		\colorbox{Black}{%
			\parbox{6em}{%
			\hfill\color{White}#2}} \par  % Duration
		\noindent \textit{#3} \par        % School
		\noindent\hangindent=2em\hangafter=0 \small #4 % Description
		\normalsize \par}

\newcommand{\WorkEntry}[4]{				  % Same as \EducationEntry
		\noindent \textbf{#1} \hfill      % Jobname
		\colorbox{Black}{\color{White}#2} \par  % Duration
		\noindent \textit{#3} \par              % Company
		\noindent\hangindent=2em\hangafter=0 \small #4 % Description
		\normalsize \par}

%%% Begin Document
%%% ------------------------------------------------------------
\begin{document}
% you can upload a photo and include it here...
%\begin{wrapfigure}{l}{0.5\textwidth}
%	\vspace*{-2em}
%		\includegraphics[width=0.15\textwidth]{photo}
%\end{wrapfigure}

\MyName{Tijl Jappens}
\MySlogan{Curriculum Vitae}

\sepspace

%%% Personal details
%%% ------------------------------------------------------------
\NewPart{Personal data}{}

\PersonalEntry{Birthdate}{17 of december, 1994}
\PersonalEntry{Adres}{Celestijnenlaan 23a, 3001 Heverlee (Leuven), Belgium}
\PersonalEntry{GSM}{+32487686456}
\PersonalEntry{Mail}{\url{tijl.jappens@hotmail.com}}
\PersonalEntry{LinkedIn}{\url{https://www.linkedin.com/in/tijl-jappens-133721295/}}

%%% Education
%%% ------------------------------------------------------------
\NewPart{Education}{}

\EducationEntry{PhD}{2019-current}{KU Leuven}{At the institute for theoretical physics. On topological phases of quantum matter. I have two manuscripts. \cite{bachmann2022classification} is under review and close to acceptance and \cite{jappens2023spt} is accepted. Both where sent to communications in mathematical physics. A third manuscript is underway.}

\EducationEntry{Master of Physics}{2016-2019}{KU Leuven}{Option research with some courses from pure mathematics.\\I had a thesis at the institute for theoretical physics. The topic was ``topological frequency conversion in strongly driven quantum systems''.\\Graduated with distinction.}
\sepspace

\EducationEntry{Bachelor of Physics}{2014-2017}{KU Leuven}{Minor mathematics.}

\NewPart{Skills}{}

\SkillsEntry{Languages}{Dutch (native language)}
\SkillsEntry{}{English (fluent)}
\SkillsEntry{}{French (limited)}

\SkillsEntry{Mathematics}{Algebra (group theory, Von Neumann algebras, $C^*$-algebras, representation theory (both for algebras and for groups)), Analyses (complex analyses, spectral theory, measure theory, numerical analyses, differential equations), Algebraic topology (homotopy, homology and cohomology groups) and many more.}

\SkillsEntry{Software}{\textsc{Python} (\textsc{Numpy}, \textsc{Scipy}, \textsc{Sympy}, \textsc{QuTip} and many others), \textsc{Git}, \textsc{Java} (\textsc{Android studio}), C\#, \textsc{Matlab}, \LaTeX}

%%% Hobby's
%%% ------------------------------------------------------------
\NewPart{Hobbies}{}

\SkillsEntry{Youth work}{Day care during the summer holidays (5 years)}
\SkillsEntry{}{Scout leader (2 years)}
\SkillsEntry{Presidium}{I was in the presidium of a student union (Wina) for two years, one of which I was a representative for the internationals of Scientica (the umbrella of the science unions of KuLeuven).}

\clearpage
\bibliography{Bibliography}
\bibliographystyle{plain}
\end{document}
